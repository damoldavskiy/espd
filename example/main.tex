\documentclass[techtask]{espd}
\usepackage[russian]{babel}
\usepackage{multirow}
\usepackage{array}

\bibliographystyle{gost2008}

\manager{А.В. Поповкин}
\managerrank{Приглашенный преподаватель\\департамента программной инженерии факультета компьютерных наук}

\author{Д.А. Молдавский}
\authorrank{студент группы БПИ183}

\title{Кроссплатформенный облачный\\текстовый редактор <<Notepad.Online>>}
\code{RU.17701729-04.03-01}

\city{Москва}
\year{2019}

\begin{document}

\annotation
Техническое задание – это основной документ, оговаривающий набор требований и порядок создания программного продукта, в соответствии с которым производится разработка программы, ее тестирование и приемка.

Настоящее Техническое задание на разработку «Кроссплатформенного облачного текстового редактора "Notepad.Online"» содержит следующие разделы: «Введение», «Основания для разработки», «Назначение разработки», «Требования к программе», «Требования к программной документации», «Технико-экономические показатели», «Стадии и этапы разработки», «Порядок контроля и приемки» приложения.

В разделе «Введение» указано наименование и краткая характеристика области применения разработки.

В разделе «Основания для разработки» указан документ, на основании которого ведется разработка и наименование темы разработки.

В разделе «Назначение разработки» указано функциональное и эксплуатационное назначение программного продукта.

В разделе «Требования к программе» указаны требования по функционалу и надежности к разрабатываемому продукту, а также условия, накладываемые на технические и информационные средства, в условиях которых предполагается использование программного продукта.

В разделе «Требования к программной документации» указан предварительный состав программной документации.

В разделе «Технико-экономические показатели» указана предполагаемая потребность, ориентировочная экономическую эффективность, а также экономические преимущества разработки по сравнению с аналогами.

В разделе «Стадии и этапы разработки» содержится информация о необходимых стадиях разработки и их содержание, а также сроки разработки и исполнители.

В разделе «Порядок контроля и приемки» указаны виды испытаний и общие требования к приемке работы.

Настоящее Техническое задание удовлетворяет требованиям ГОСТ 19.201-78~\cite{espd201}. Перед прочтением данного документа рекомендуется ознакомиться с используемой терминологией. Определения могут быть уточнены в контексте документации.

\tableofcontents

\section{Введение}

\subsection{Наименование программы}
\paragraph{Наименование программы на русском языке}
Кроссплатформенный облачный текстовый редактор "Notepad.Online".

\paragraph{Наименование программы на английском языке}
Cross-platform Cloud Text Editor «Notepad.Online».

\subsection{Краткая характеристика области применения}
На момент написания данного документа на рынке представлено значительное количество программ для хранения текстовых заметок в облачном хранилище, однако ни одна из них не является, на взгляд составителя ТЗ, одновременно, с одной стороны, достаточно удобным и функциональным редактором на платформах ПК и Android, и, с другой, легковесным приложением, позволяющим делать небольшие заметки достаточно быстро и не требовательного к ресурсам устройства.

Целью разработки данного приложения является создание программного комплекса, составом которого будет совокупность текстовых редакторов для операционных систем Windows (далее – десктопное приложение) и Android (далее – мобильное приложение), связанных с одной базой данных и дополнительными системами для хранения информации о пользователях и их файлов, а также их обработки (далее – облачное хранилище). Программный комплекс должен позволять иметь возможность хранения текстовых файлов зарегистрированных пользователей в облачном хранилище, а также работать с ними с обоих ОС.

\section{Основания для разработки}

\subsection{Документ, на основании которого ведется разработка}
Основанием для разработки данного приложения является приказ декана Факультета компьютерных наук И.В. Аржанцева № 2.3-02/1012-0 2 от 10.12.2018 «Об утверждении тем и руководителей курсовых работ студентов образовательной программы "Программная инженерия"».

\subsection{Наименование темы разработки}
Наименование темы разработки – «Кроссплатформенный облачный текстовый редактор "Notepad.Online"».

Условное обозначение темы разработки – «Notepad.Online».

Программа выполняется в рамках темы курсовой работы в соответствии с учебным планом подготовки бакалавров по направлению 09.03.04 «Программная инженерия» Национального исследовательского университета «Высшая школа экономики», факультет компьютерных наук, департамент программной инженерии.

\section{Назначение разработки}

\subsection{Функциональное назначение}
Программный комплекс должен содержать следующие функции:

\begin{enumerate}
\item Регистрация и авторизация пользователей;
\item Авторизация, в том числе автоматическая авторизация при запуске программы;
\item Хранение настроек программы на устройстве;
\item Создание и работа с текстовыми файлами;
\item Хранение данных о файлах и пользователях в облачном хранилище;
\item Синхронизация файлов с облачным хранилищем.
\end{enumerate}

В десктопной версии пользователю должны быть предоставлены все основные средства для работы с текстовыми файлами: выделение, вставка, удаление, поиск, отмена операций. Также должна присутствовать возможность использования облачного хранилища: регистрация, вход, отправка файла на сервер, синхронизация с клиентом.

Мобильная версия должна быть в большей степени ориентирована на работу с облачным хранилищем. Так, должны поддерживаться все те же операции работы с синхронизацией файлов. Интерфейс работы с файлом должен поддерживать его изменение посредством встроенных в Android инструментов. Как и в десктопной версии, должна присутствовать возможность настройки программы и работы с аккаунтом.

\subsection{Эксплуатационное назначение}
Создание кроссплатформенного текстового редактора с синхронизацией является востребованной задачей на сегодняшний день, так как возможность быстрого создания файлов и доступ к ним с любого устройства – это одна из базовых потребностей во многих сферах деятельности (например, педагогическая и управленческая).

Благодаря простоте интерфейса десктопной версии Notepad.Online может быть использован для замены стандартного текстового редактора «Блокнот» ОС Windows. В таком случае пользователь приобретает возможность с минимальным изменением привычек начать пользоваться облачным хранилищем и другим функционалом разрабатываемого приложения.

Плагин «Сниппеты» для десктопной версии позволит использовать Notepad.Online для эффективной записи информации, в особенности исходный код на языках программирования.

\section{Требования к программе}

\subsection{Требования к функциональным характеристикам}
Облачное хранилище должно обладать возможностью хранения информации о пользователях и сохраненных ими данных, пароли должны храниться в зашифрованном виде, а доступ к данным должен быть реализован через интерфейс токенов. Регистрация в системе должна производиться с использованием привязки к почтовому ящику пользователя, для этого при создании аккаунта пользователь должен получать письмо на свой электронный ящик с четырехзначным кодом подтверждения регистрации. Также должна быть реализована возможность изменения пароля, а также его восстановления, путем подтверждения действия через электронный ящик. База данных и все дополнительный функции должны быть реализованы на сервисе облачных вычислений Azure.

Мобильное приложение должно обладать интерфейсом для регистрации и авторизации пользователя, функцией автоматической авторизации при запуске приложения, создания, удаления, редактирования файлов, их синхронизации с сервером. Также должна присутствовать возможность создания новых файлов с помощью анализа текста на фотографии. Интерфейс пользователя должен состоять из страницы входа, главной страницы приложения (с отображением списка текущих файлов и кнопок, отвечающих за создание и синхронизацию), окна редактирования файла и других необходимых окон.

Десктопное приложение должно предоставлять пользователю возможность полноценной работы с текстовыми файлами (выделение, вставка, удаление, поиск, отмена операций). Должна присутствовать возможность настройки интерфейса пользователя (изменение цвета окна, цвета текста, параметры приложения), также должна присутствовать возможность подключения плагинов для расширения функционала. Функциональный модуль для работы синхронизации с базой данных, авторизацией и регистрацией должен быть оформлен в виде подключаемого модуля. Интерфейс пользователя должен представлять из себя окно редактирования текущего файла и меню (см. справочное приложение). Меню должно быть представлено из классических для текстовых редакторов пунктов (таких как «Файл», «Правка», «Вид»), а также пункта для работы с подключаемыми модулями.

В качестве дополнительного модуля для десктопной версии должен быть реализован плагин «Сниппеты». Данный плагин должен, используя возможности расширяемости приложения и возможности .NET Framework, позволять пользователю создавать и настраивать сниппеты. Пользователь должен иметь возможность использовать в сниппетах вставки на языке Python, устанавливать флаги, влияющие на возможность срабатывания сниппета при редактировании файла.

Общим требованием ко всему программному комплексу и плагинам является глобализация. Интерфейс и все сообщения пользователю должны быть переведены на русский и английский языки, при этом язык программы должен автоматически определяться при запуске и зависеть от языка окружения приложения.

\subsection{Требования к надежности}
Для обеспечения корректности работы программы должна быть предусмотрена проверка входных данных на всех этапах использования программного комплекса. При этом, проверка данных должна происходить как на стороне клиента, так и на стороне сервера. При отсутствии интернет-соединения пользователь должен быть уведомлен об этом, а процессы синхронизации должны быть приостановлены до возобновления интернет-соединения.

Для обеспечения безопасности данных пользователей на стороне БД должно быть использовано прозрачное шифрование данных. Алгоритм восстановления пароля по коду должен быть реализован таким образом, чтобы задача взлома аккаунта путем перебора кодов восстановления была алгоритмически сложной, невыполнимой или выполнимой с малой вероятностью задачей в пределах месяца.

\subsection{Условия эксплуатации}
Климатические условия эксплуатации программы определяются климатическими условиями эксплуатации оборудования, используемого для хранения и запуска приложений, т.е. компьютеров и телефонов непромышленного исполнения. Таким образом, должны выполняться следующие условия:

\begin{enumerate}
\item Влажность – не более 80\%;
\item Температура – от 10$^\circ$ C до 30$^\circ$ C;
\item Атмосферное давление от 630 до 800 мм. ртутного столбца;
\item Отсутствие газообразных кислот и коррозийных веществ в воздухе;
\item Запыленность не более 0.75 мг/м$^3$.
\end{enumerate}

Для использования программного комплекса пользователь должен иметь опыт работы с операционными системами Windows и Android. Кроме этого, необходимыми являются навыки работы с текстовыми редакторами и веб-приложениями; также пользователь должен иметь базовые представления о принципе работы данного приложения.

\subsection{Требования к составу и параметрам технических средств}
Для работы десктопного приложения необходим компьютер, характеристики которого позволяют производить стабильную работу с операционной системой Windows 7 любой редакции и всеми ее компонентами:

\begin{enumerate}
\item Процессор с тактовой частотой не ниже 1 ГГц;
\item Оперативная память не меньше 2 Гб;
\item Свободное место на жестком диске 100 Мб или больше;
\item Наличие мышки или сенсорной панели, клавиатуры, монитора и хотя бы одного USB разъема.
\end{enumerate}

Для работы мобильного приложения необходим смартфон, характеристики которого позволяют работать и пользоваться всеми основными функциями операционной системы Android не ниже версии 4.4: 

\begin{enumerate}
\item Процессор с тактовой частотой не ниже 1 ГГц;
\item Оперативная память не меньше 512 Мб;
\item Свободное место на встроенной памяти не меньше 60 Мб;
\item Сенсорный экран с диагональю не меньше 4 дюймов;
\item Наличие камеры.
\end{enumerate}

Кроме того, для работоспособности механизма синхронизаций необходим бесперебойный доступ к сети Интернет с обоих устройств.

Вышеописанные условия обусловлены функциональностью программы, примерным размером архивов и исполняемых файлов, а также минимальными системными требованиями целевых операционных систем.

\subsection{Требования к информационной и программной совместимости}
Для работы десктопного приложения необходима установленная операционная система Windows 7 или более поздний выпуск Windows, а также установленный распространяемый пакет .NET Framework версии 4.6.1 или выше.

Для работы мобильного приложения необходима установленная операционная система Android 4.4 (API 19) или выше.

\subsection{Требования к маркировке и упаковке}
В комплект поставки программы входит USB-флеш-накопитель, на котором хранятся следующие элементы программы:

\begin{enumerate}
\item Исполняемый файл формата .exe (десктопное приложение). Файл должен обладать метаинформацией о программе;
\item Необходимые библиотеки для десктопного приложения в формате .dll;
\item Установочный файл формата .apk для работы мобильного приложения; файл должен быть подписан ключом с идентификатором «com.dmsoft.notepadonline».
\end{enumerate}

\subsection{Требования к транспортированию и хранению}
Требования к транспортировке и хранению программных документов являются стандартными и должны соответствовать общим требованиям хранения и транспортировки печатной продукции:

\begin{enumerate}
\item В помещении для хранения печатной продукции допустимы температура воздуха от 10$^\circ$С до 30$^\circ$С и относительная влажность воздуха от 30\% до 60\%;
\item Документацию хранят и используют на расстоянии не менее 0.5 от источников тепла и влаги. Не допускается хранение печатной продукции в помещениях, где находятся агрессивные агенты – растворители, спирт, бензин;
\item Не допускается попадание на документацию агрессивных агентов;
\item Транспортировка производится в специальных контейнерах с применением мер по предотвращению деформации документов внутри контейнеров, а также проникновения влаги, вредных газов, пыли, солнечных лучей и образованию конденсата внутри контейнеров;
\item Программные документы, предоставляемые в печатном виде, должны соответствовать общим правилам учета и хранения программных документов, предусмотренных стандартами Единой системы программной документации и соответствовать требованиям ГОСТ 19.602-78~\cite{espd602}.
\end{enumerate}

\section{Требования к программной документации}

\subsection{Предварительный состав программной документации}\label{subsection:documentation}
«Кроссплатформенный облачный текстовый редактор "Notepad.Online"». Техническое задание (ГОСТ 19.201-78~\cite{espd201})

«Кроссплатформенный облачный текстовый редактор "Notepad.Online"». Программа и методика испытаний (ГОСТ 19.301-78~\cite{espd301})

«Кроссплатформенный облачный текстовый редактор "Notepad.Online"». Пояснительная записка (ГОСТ 19.404-79~\cite{espd404})

«Кроссплатформенный облачный текстовый редактор "Notepad.Online"». Руководство оператора (ГОСТ 19.505-79~\cite{espd505})

«Кроссплатформенный облачный текстовый редактор "Notepad.Online"». Текст программы (ГОСТ 19.401-78~\cite{espd401})

\subsection{Специальные требования к программной документации}
\begin{enumerate}
\item Пояснительная записка должна быть загружена через информационно-образовательную среду НИУ ВШЭ LMS в систему «Антиплагиат», допустимый процент заимствования – 40\%;
\item Техническая документация, программа, исходные коды и презентация загружаются в LMS одним архивом в формате .zip.
\item Согласованная и утвержденная документация сдается в печатном виде в учебный офис образовательной программы 09.03.04 «Программная инженерия» НИУ ВШЭ.
\end{enumerate}

Сроки сдач всех указанных документов и данных определяются приказом декана Факультета компьютерных наук И.В. Аржанцева № 2.3-02/1012-0 2 от 10.12.2018. 

\section{Технико-экономические показатели}

\subsection{Предполагаемая потребность}
Конечный программный комплекс должен предоставлять возможность быстро создавать текстовые файлы, а также иметь возможность работы с ними как на мобильном устройстве, так и на персональном компьютере. Таким образом, целевой аудиторией программы являются люди, чья деятельность предполагает частое сохранение разного рода текстовой информации на разных устройствах: учащиеся школ, вузов, предприниматели и другие.

\subsection{Ориентировочная экономическая эффективность}
Расчет экономической эффективности в данной работе не предусмотрен.

\subsection{Экономические приемущества разработки по сравнению с отечественными и заруюежными аналогами}
Предполагается, что использование данного продукта позволит сделать хранение и редактирование информации более удобным, чем при использовании имеющихся на рынке аналогов - преимуществом данного продукта должно стать сочетание функциональности текстового редактора и простота облачного хранения данных.

\section{Стадии и этапы разработки}

\subsection{Необходимые стадии разработки, этапы и содержание работ}
\noindent\begin{tabular}{|>{\raggedright}p{50mm}|>{\raggedright}p{55mm}|>{\raggedright\arraybackslash}p{60mm}|}
\hline
Стадии & Этапы работ & Содержание работ \\ \hline
\multirow[t]{8}{=}{1. Техническое задание} & \multirow[t]{2}{=}{Обоснование необходимости разработки программы} & Постановка исходных материалов \\ \cline{3-3}
& & Сбор исходных материалов \\ \cline{2-3}
& \multirow[t]{3}{=}{Научно-исследовательские работы} & Определение структуры входных и выходных данных \\ \cline{3-3}
& & Определение требований к техническим средствам \\ \cline{3-3}
& & Обоснование принципиальной возможности решения поставленной задачи \\ \cline{2-3}
& \multirow[t]{3}{=}{Разработка и утверждение технического задания} & Определение требований к программе \\ \cline{3-3}
& & Определение стадий, этапов и сроков разработки программы и документации на нее \\ \cline{3-3}
& & Согласование и утверждение технического задания \\ \hline
\multirow[t]{3}{=}{2. Рабочий проект} & Разработка программы & Программирование и отладка программы \\ \cline{2-3}
& Разработка программной документации & Разработка программных документов в соответствии с требованиями ГОСТ 19.101-77~\cite{espd101} \\ \cline{2-3}
& Испытания программы & Разработка, согласование и утверждение порядка и методики испытаний \\ \hline
\multirow[t]{4}{=}{3. Внедрение} & \multirow[t]{4}{=}{Подготовка и передача программы} & Подготовка программы и программной документации для презентации и защиты \\ \cline{3-3}
& & Утверждение дня защиты программы \\ \cline{3-3}
& & Презентация программного продукта \\ \cline{3-3}
& & Передача программы и программной документации в архив НИУ ВШЭ \\ \hline
\end{tabular}

\subsection{Сроки разработки и исполнители}
Крайний срок разработки – 22 мая 2019 года. Исполнитель – Молдавский Денис Александрович, студент группы БПИ183.

\section{Порядок контроля и приемки}

\subsection{Виды испытаний}
В рамках испытаний проводится проверка на корректность работы и соответствие требованиям данного Технического задания каждого функционального элемента программного комплекса. Испытания должны включать как тесты каждой из создаваемых систем по отдельности, так и тесты их функционирования вместе, в соответствии с документом «Программа и методика испытаний».

\subsection{Общие требования к приемке работы}
Прием работы происходит в виде защиты выполненного проекта комиссии, состоящей из преподавателей Департамента программной инженерии, в утверждённые приказом декана ФКН сроки. Прием происходит при полной работоспособности программного комплекса, наличии всей документации, указанной в подразделе~\ref{subsection:documentation} и соответствующего требованиям настоящего Технического задания.

\bibliography{espd}

\begin{terms}
\term{Python}{интерпретируемый язык программирования высокого уровня}
\term{Десктопное приложение}{совокупность исполняемого файла, ресурсов и динамически загружаемых библиотек, которые содержат программу и предназначены для ее запуска в операционной системе Windows}
\term{Мобильное приложение}{программа, хранимая в виде файла с расширением .apk для запуска в операционной среде Android}
\term{Облачное хранилище}{сервисная модель, в которой данные хранятся, управляются, резервируются удаленно и предоставляются пользователям по сети}
\term{Окружение приложения}{контекст, в котором выполняется программа, т.е. операционная система и другие элементы, которые потенциально могут влиять на работу приложения}
\term{Отмена операций}{часть инструментального функционала большинства текстовых редакторов, функция, позволяющая отменить внесенные пользователем изменения в документ, такие как вставка символов, удаление символов, влияющие на содержимое встроенные функции программы}
\term{Плагин (расширение)}{дополнительный модуль программы, подключаемый к основной программе во время выполнения (Runtime) и предназначенный для расширения функционала основной программы}
\term{Прозрачное шифрование базы данных}{функция, выполняющая шифрование и дешифрование ввода-вывода в реальном времени для файлов данных и журналов. Ключ шифрования при этом хранится в загрузочной записи базы данных}
\term{Сниппет}{функция, заменяющая при определенных условиях подстроку в текстовом поле на другую строку, заранее заданную или сгенерированную по заранее определенным правилам}
\end{terms}

\begin{abbreviations}
\term{API}{Application Programming Interface}
\term{USB}{Universal Serial Bus}
\term{ОС}{операционная система}
\term{ПК}{персональный компьютер}
\end{abbreviations}

\attachment{Справочное}{Окно программы <<Блокнот>> ОС Windows 10}
\image{notepad}

\end{document}
