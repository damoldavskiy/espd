\documentclass[opermanual]{espd}
\usepackage[russian]{babel}

\bibliographystyle{gost2008}

\manager{А.В. Поповкин}
\managerrank{Приглашенный преподаватель\\департамента программной инженерии факультета компьютерных наук}

\author{Д.А. Молдавский}
\authorrank{студент группы БПИ183}

\title{Кроссплатформенный облачный\\текстовый редактор <<Notepad.Online>>}
\code{RU.17701729-04.03-01}

\city{Москва}
\year{2019}

\begin{document}

\annotation
Руководство оператора -- это документ, содержащий сведения для обеспечения корректного использования программного комплекса оператором.

Настоящее Руководство оператора программы «Кроссплатформенный облачный текстовый редактор "Notepad.Online"» содержит следующие разделы: «Назначение программы», «Условия выполнения программы», «Выполнение программы», «Сообщения оператору».

В разделе «Назначение программы» указано эксплуатационное назначение и краткая характеристика функциональности программы.

В разделе «Условия выполнения программы» указаны требования к аппаратным и программным средствам, используемым оператором.

В разделе «Выполнение программы» указана последовательность действий, ведущая к использованию основных функциональных возможностей программного комплекса и корректному завершению программы.

В разделе «Сообщения оператору» указаны основные сообщения, которые может получить пользователь при использовании программы. По причине того, что большинство сообщений программы генерируется во время исполнения, указаны лишь основные из причин их появления.

Настоящее Руководство оператора удовлетворяет требованиям ГОСТ 19.505-79~\cite{espd505}. Перед прочтением данного документа рекомендуется ознакомиться с используемой терминологией. Определения могут быть уточнены в контексте документации.

\tableofcontents

\section{Назначение программы}

Программный комплекс предназначен для редактирования текстовых файлов в операционных системах Windows и Android с хранением файлов в облачном хранилище. При этом десктопная версия повторяет и расширяет функционал стандартного приложения «Блокнот». Функция расширения функционала путем подключения плагинов позволяет разрабатывать и использовать произвольные алгоритмы с использованием технологии .NET Framework. Встроенный плагин «Облачное хранилище» предоставляет простой интерфейс для работы с облаком, а встроенный плагин «Сниппеты» включает мощный механизм сниппетов, который может использоваться для решения широкого круга задач, связанной с повторным вводом однотипным блоков текста.

Мобильная версия направлена на работу с файлами, хранящимися в облаке. Кроме того, представлен специальный функционал быстрой записи информации -- функция распознавания текста на фотографии.

\section{Условия выполнения программы}

\subsection{Требования к аппаратным средствам}
\paragraph{Требования к аппаратным средствам для десктопного приложения}
Для работы десктопного приложения необходим компьютер, характеристики которого позволяют производить стабильную работу с операционной системой Windows 7 любой редакции и всеми ее компонентами:

\begin{enumerate}
\item Процессор с тактовой частотой не ниже 1 ГГц;
\item Оперативная память не меньше 2 Гб;
\item Свободное место на жестком диске 100 Мб или больше;
\item Наличие мышки или сенсорной панели, клавиатуры, монитора и хотя бы одного USB разъема.
\end{enumerate}

\paragraph{Требования к аппаратным средствам для мобильного приложения}
Для работы мобильного приложения необходим смартфон, характеристики которого позволяют работать и пользоваться всеми основными функциями операционной системы Android не ниже версии 4.4: 

\begin{enumerate}
\item Процессор с тактовой частотой не ниже 1 ГГц;
\item Оперативная память не меньше 512 Мб;
\item Свободное место на встроенной памяти не меньше 60 Мб;
\item Сенсорный экран с диагональю не меньше 4 дюймов;
\item Наличие камеры.
\end{enumerate}

\paragraph{Общие требования к программным средствам}
Для работоспособности механизма синхронизаций необходим бесперебойный доступ к сети Интернет с обоих устройств.

\subsection{Требования к программным средствам}
\paragraph{Требования к программным средствам для десктопного приложения}
Для работы десктопного приложения необходима установленная операционная система Windows 7 или более поздний выпуск Windows, а также установленный распространяемый пакет .NET Framework версии 4.6.1 или выше.

\paragraph{Требования к программным средствам для мобильного приложения}
Для работы мобильного приложения необходима установленная операционная система Android 4.4 (API 19) или выше.

\section{Выполнение программы}

Интерфейс десктопной версии Notepad.Online традиционен для текстовых редакторов. Далее будет описана последовательность действия, иллюстрирующая основные функции программы. После запуска программы следует напечатать текст. Программа примет следующий вид (\ref{window}).

\illustration[Окно программы]{MainWindow}[window]

Рассмотрим основные пункты меню. Пункты, отвечающие за операции, которые в данный момент по какой-либо причине выполнить невозможно, окрашиваются серым. В пункте меню «Файл» имеются инструменты для открытия и сохранения текстовых файлов. В пункте меню «Правка» имеются инструменты для отмены операций, операций с буфером (копирование в буфер, копирование в буфер с удалением текста (вырезание), вставка из буфера), также имеются пункты «Найти» и «Заменить». Первая операция предназначена для нахождения подстрок в тексте, вторая предназначена для нахождения и замены. Рассмотрим интерфейс первой (\ref{find}).

\illustration[Окно функции <<Найти>>]{FindWindow}[find]

Первый пункт выбора предназначен для указания функции игнорировать регистр входной строки. Пункт «Рег. выражение» при включении указывает функции использовать входную строку как шаблон для поиска на языке регулярных выражений. Направление поиска указывает направление для распознавания подстроки, начиная с позиции курсора. При вводе некоторой подстроки введенного текста в данное поле и нажатии кнопки «Найти след.» фокус перейдет на главное окно, а найденная строка будет выделена. Если строка не будет найдена, всплывет соответствующее сообщение.
Перейдем в пункт «Вид», выберем опцию «Настройки». В открывшемся окне в списке слева выберем пункт «Редактор». Приложение примет следующий вид (\ref{settings}).

\illustration[Окно настроек]{SettingsWindow}[settings]

В данном окне можно выбрать характеристики шрифта, которые используются для отображения текста в окне редактора. Изменим размер шрифта с 14 на 16 и нажмем «Сохранить». Всплывет окно, указывающее, что настройки были изменены. Это окно можно закрыть. Шрифт редактора изменится (\ref{aftersettings}). Далее будет рассматриваться пустое поле редактора.

\illustration[Окно программы после изменения шрифта]{WindowAfterSettings}[aftersettings]

Также в пункте «Вид» можно выбрать «Перевод строк». В этом случае при отображении строк, которые не умещаются в ширину окна редактора, эти строки будут разделены на несколько других строк.
В пункте меню «Расширения» выберем пункт «Менеджер». Поочередно выберем и включим оба встроенных по умолчанию расширения (\ref{manager}).

\illustration[Менеджер расширений]{ExtensionManager}[manager]

После включения расширений можно убедиться, что расширение «Сниппеты» работает и функционирует. Вместе с программой в данном расширении идут настройки, предназначенные для высокоэффективной записи текста (в основном формул) на языке LaTeX. Проверим это, введя в главное окно программы текст «begdocument». Окно программы примет следующий вид (\ref{snippet}).

\illustration[Активация сниппета]{SnippetActivation}[snippet]

Как видно на рисунке, при вводе текста «beg» активизировался сниппет, отвечающий за начало блока. Дальнейший текст был распознан мультикареткой. Теперь нужно нажать Tab, каретка перейдет в пространство между командами с отступом в 4 пробела. Таким образом, был использован менеджер расширений и включен плагин, который начал свою работу.
Воспользуемся плагином «Облачное хранилище». Перейдя в одноименную вкладку пункта меню «Расширения» и выбрав пункт «Параметры», нажмем на кнопку «Регистрация и выполним регистрацию (\ref{registration}).

\illustration[Регистрация]{Registration}[registration]

После подтверждения регистрации и входа следует закрыть параметры и в меню плагина выбрать пункт «Сохранить в облаке». В открывшемся окне следует ввести название файла для сохранения и подтвердить его (\ref{filename}).

\illustration[Ввод названия файла для сохранения в облаке]{FileName}[filename]

Если до этого файл был сохранен на диск компьютера, то сохранение будет автоматически выполнено с названием файла (без расширения). Если в облаке уже имеется файл с таким названием, всплывет диалоговое окно о замене файла.
Теперь следует закрыть приложение нажатием на крестик. При возникновении диалога о сохранении введенного текста на компьютере следует нажать «Нет».

Теперь откроем мобильное приложение (предварительно его необходимо установить, это делается классическим для Android-приложений способом, более подробно этот процесс описан в Программе и методике испытаний). Теперь следует ввести почтовый ящик и пароль, под которыми проходила регистрация и подтвердить вход (\ref{signin}).

\illustration[Страница входа][][0.4]{SignInPage}[signin]

После подтверждения входа (при успешности операции) будет открыто окно со списком текущих файлов (\ref{explorer}).

\illustration[Страница с файлами][][0.4]{ExplorerPage}[explorer]

Теперь следует нажать на созданный файл и ввести новою строку, после чего нужно нажать кнопку «Сохранить» во всплывающем меню (\ref{file}).

\illustration[Страница файла][][0.4]{FilePage}[file]

Файл был изменен. Проверим это в десктопной версии. Перезапустим программу (можно также обновить список файлов в соответствующем пункте меню плагина) или откроем ее, если программа была закрыта. Откроем искомый файл в папке «NotepadOnline» стандартного каталога документов пользователя. Ее содержимое изменится, подтверждая корректность предыдущих действий (\ref{updated}).

\illustration[Обновленный файл]{UpdatedFile}[updated]

Теперь можно закрыть приложение на всех устройствах. Основной функционал был рассмотрен и использован, программа корректно завершена.

\section{Сообщения оператору}

\subsection{Тексты возможных сообщений десктопной версии}
\term{Необходимо войти в аккаунт}{операция не была выполнена, так как вход не выполнен. Необходимо открыть параметры плагина и войти в аккаунт}
\term{Откройте файл из папки облака для работы с ним}{операция не была выполнена, так как открытый файл не отражает связь с синхронизируемым файлом. Необходимо перейти в директорию документов пользователя и в папке «NotepadOnline» выбрать нужный файл}
\term{Файл удален}{операция удаления файла прошла без ошибок}
\term{Все файлы обновлены}{синхронизация файлов прошла без ошибок}
\term{Файл сохранен в облаке}{отправка файла в облако прошла без ошибок}
\term{Файл не сохранен}{при отправке файла на сервер произошла ошибка}
\term{Вход выполнен}{операция входа прошла без ошибок}
\term{Директория не существует}{указанный каталог не существует. Необходимо проверить корректность введенного пути}
\term{Пароли не совпадают}{пароль подтверждения не эквивалентен паролю. Необходимо проверить правильность повторного ввода пароля}
\term{Сначала пройдите регистрацию}{необходимо пройти регистрацию}
\term{Невозможно загрузить файл}{открытие файла не завершилось успехом}
\term{Некорректный аргумент}{указанное значение синтаксически неверно}
\term{Нет совпадений}{поиск указанной подстроки в тексте не дал результатов}

\subsection{Тексты возможных сообщений мобильной версии}
\term{Введите текущий пароль}{необходимо ввести действительный пароль}
\term{Повторите новый пароль, введя его в соответствующее поле}{указанные пароли не совпадают}
\term{Пароль успешно изменен}{операция смены пароля прошла без ошибок}
\term{Во время получения ключевых слов произошла ошибка}{операция получения ключевых слов прошла с ошибкой. Необходимо обратиться к разработчику}
\term{Новое название содержит недопустимые символы}{необходимо ввести название файла, которое удовлетворяет правилам наименования файла Windows}
\term{Во время получения данных файлов произошла ошибка}{необходимо проверить подключение к интернету и повторить операцию. В случае, если ошибка не исчезла, следует обратиться к разработчику}
\term{Нет разрешения выбирать фотографии (нет разрешения делать фотографии)}{необходимо перейти в настройки Android и предоставить соответствующее разрешение приложению}
\term{Галерея недоступна}{галерея Android недоступна. Необходимо проверить совместимость версии Android. Если противоречий нет, необходимо установить соответствующее приложение}
\term{Камера недоступна}{необходимо проверить наличие камеры у используемого устройства. Если камера присутствует, необходимо обратиться к разработчику}

\subsection{Общие тексты возможных сообщений}
\term{Процесс занят}{при новом запросе к базе данных еще не получен ответ на предыдущий запрос. Сообщение может возникнуть, если при отсутствии подключения несколько раз сделать запрос. Для решения необходимо дождаться ответа предыдущего запроса}
\term{Код подтверждения устарел}{срок годности указанного кода подтверждения истек. Для решения необходимо сделать запрос регистрации еще раз}
\term{Файл с данным именем уже существует}{попытка сохранить файл под уже занятым именем. Необходимо выбрать другое имя}
\term{Файл с данным именем не существует}{операция не может быть выполнена, так как указанного файла не существует. Для решения необходимо закрыть текущий файл и обновить список файлов}
\term{Код подтверждения некорректен}{код подтверждения не является четырехзначным числом. Данное сообщение возможно только при сторонних реализациях}
\term{Почтовый ящик некорректен}{указанный почтовый ящик синтаксически неверен. Для решения необходимо указать правильный адрес почты}
\term{Пароль некорректен}{Длина пароля не попадает в разрешимый диапазон. Длина пароля может быть от 4 до 40 символов}
\term{Нет подключения к серверу}{отсутствует интернет-соединение. Для решения необходимо обратиться к провайдеру интернет-услуг}
\term{Пользователь для подтверждения не найден}{в базе данных нет записи для подтверждения пользователя, так как время подтверждения истекло. Необходимо пройти регистрацию заново}
\term{Достигнут предел хранилища}{попытка загрузить файл, суммарный размер хранилища пользователя с которым больше 4 Мб. Необходимо обратиться к разработчику или удалить ненужные файлы из хранилища}
\term{Успешная операция}{операция прошла без ошибок. Дополнительных действий не требуется}
\term{Токен не существует}{токен, который был отправлен на сервер, не содержится в базе данных. Данное сообщение возможно только при сторонних реализациях}
\term{Токен устарел}{токен, который был отправлен на сервер, более не действителен. Данное сообщение возможно только при сторонних реализациях}
\term{Слишком много запросов восстановления пароля}{превышен лимит запросов на восстановление пароля. Необходимо повторить запрос через 24 часа}
\term{Пользователь с данным почтовым ящиком уже существует}{указанный почтовый адрес занят. Необходимо выбрать другой почтовый ящик или восстановить пароль}
\term{Пользователь с данными почтовым ящиком и паролем не существует}{указанная пара логин-пароль не содержится в базе данных. Соответственно, неверно введет почтовый адрес или пароль. Необходимо проверить данные и повторить запрос}
\term{Неверный код подтверждения}{указанный код подтверждения регистрации не является верным. Необходимо проверить пришедшее на указанный почтовый адрес сообщение и повторить ввод кода}

\bibliography{espd}

\begin{terms}
\term{Десктопное приложение}{совокупность исполняемого файла, ресурсов и динамически загружаемых библиотек, которые содержат программу и предназначены для ее запуска в операционной системе Windows}
\term{Мобильное приложение}{программа, хранимая в виде файла с расширением .apk для запуска в операционной среде Android}
\term{Мультикаретка}{функция текстового редактора, которая позволяет выводить напечатанный текст в двух или более местах окна редактора}
\term{Облачное хранилище}{сервисная модель, в которой данные хранятся, управляются, резервируются удаленно и предоставляются пользователям по сети}
\term{Отмена операций}{часть инструментального функционала большинства текстовых редакторов, функция, позволяющая отменить внесенные пользователем изменения в документ, такие как вставка символов, удаление символов, влияющие на содержимое встроенные функции программы}
\term{Плагин (расширение)}{дополнительный модуль программы, подключаемый к основной программе во время выполнения (Runtime) и предназначенный для расширения функционала основной программы}
\term{Сниппет}{функция, заменяющая при определенных условиях подстроку в текстовом поле на другую строку, заранее заданную или сгенерированную по заранее определенным правилам}
\term{LaTeX}{язык разметки документов общего назначения}
\end{terms}

\end{document}
